\section{Resultat och analys / Designprocessen}
I detta kapitel ska du beskriva dina resultat och analysera dem, dvs. det du har kommit fram till i ditt arbete. Beroende på vad du gjort kan detta kapitel se ut på olika sätt. Det är viktigt att du redovisar resultaten på ett stringent sätt och att du presenterar fakta utan att lägga in personliga synpunkter eller värderingar. Det gör du i stället i det avslutande diskussionskapitlet. Det är viktigt att det framgår av texten vad som är resultat och vad som är din analys av resultaten. Den kan vara en fördel att använda frågeställningarna som rubriker för att beskriva resultaten för att vara säker på att du verkligen besvarar dessa.\\ \\
Du som arbetar med design av produkter/system ska i detta kapitel redogöra för själva designprocessen, dvs. du beskriver här de olika stegen i designprocessen, vilka val du gjort och motiverar dina val. Här lägger du in relevanta skisser, modeller och figurer \index{figurer} så att läsaren får en god bild av hur du kommit fram till din slutprodukt. I ett designarbete är det viktigaste att läsaren förstår hur du kom fram till just ditt designförslag eller din slutprodukt. Här ska du också analysera slutproduktens funktioner utifrån vad du tänkt dig från början. Koppla ihop resultatet med syfte för att tydliggöra på vilket sätt du löst problemet utifrån dina frågeställningar. Använd illustrationer för att peka på särskilt intressanta områden/detaljer som är viktiga för produkten.\\ \\
Diagram och tabeller används för att underlätta tolkning av texten. Figur nedan är ett exempel där korsreferens har använts vilket möjliggör att flytta en figur. Detta gäller hela rapporten, där en figur kan underlätta förståelsen är det motiverat att ha en figur med. I designprocessen \index{designprocessen} är det förutsättning att använda skisser/modeller/figurer etc. för att läsaren ska få en uppfattning om hur arbetet fortskridit. Det är viktigt att det finns hänvisning till alla figurer och tabeller i texten i de fall du inte gjort dem själv. Observera att Figurtext skrivs under figuren och avslutas med punkt medan Tabelltext skrivs över respektive tabell och avslutas utan punkt. Tänk på att alltid introducera och hänvisa till tabeller och figurer i den löpande texten.