\section{Metod och genomförande}
Här börjar du med att motivera och beskriva vilken typ av studie \index{studie} du gjort, t ex en fallstudie eller en experimentell studie samt vilka specifika metoder du valt och vad de innebär, t ex intervju, enkät, eller observation. Hänvisa till referenser bakom de olika metoderna som i teorikapitlet. Därefter redogör du för hur du utfört arbetet, dvs. hur du gått tillväga för att kunna besvara dina frågeställningar och uppnå syftet. Hit hör t ex. hur du genomfört intervjuer, utrustning du har använt, samt beskrivningar av försök som du gjort. Redogör för hur du har samlat in och bearbetat data. Var noggrann i din beskrivning eftersom den påverkar bedömningen av validitet \index{validitet} (giltighet) och reliabilitet (trovärdighet) i examensarbetet!\\ \\
För dig som arbetar med design av produkter/system ska detta avsnitt inbegripa en kort och övergripande beskrivning av de olika faserna i din arbetsprocess och hur du använt olika designverktyg/hjälpmedel under designprocessen, t ex designbrief och funktionsanalys. Ange även här referenser. Under resultatkapitlet gör du sedan en mer detaljerad beskrivning av hur du gått tillväga för att komma fram till en slutprodukt.