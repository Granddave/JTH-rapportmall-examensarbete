Sammanfattningen är en kort beskrivning av rapportens innehåll och bör omfatta högst en A4-sida. Det ska vara en ”minirapport” på högst en A4-sida. Den ska innehålla en kort bakgrund, syfte, frågeställningar, metod, hur arbetet genomförts, resultat och slutsatser. Tyngdpunkten bör ligga på resultatet. Observera att sammanfattningen absolut inte ska vara en upprepning av innehållsförteckningen.\\ \\
Hänvisningar till löptexten ska inte göras.\\ \\
Syftet med sammanfattningen är att läsaren snabbt ska få en uppfattning om arbetets förutsättningar och resultat. Det är därför viktigt att sammanfattningen innehåller så konkreta uppgifter som möjligt. Det är viktigt att du lägger mycket tid på att formulera dig eftersom sammanfattningen ofta är det som avgör om rapporten bedöms som värd att läsa eller inte. Sammanfattningen är det som man skriver sist, dvs. när rapporten i övrigt är klar.